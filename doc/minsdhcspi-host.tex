\documentclass[oneside]{article}


\errorcontextlines 10000

\usepackage[utf8]{inputenc}
\usepackage[english]{babel}
\usepackage{graphicx}
\usepackage{minted}   % VHDL Code
\usepackage[numbers]{natbib}   % Biblio

\usepackage[a4paper,left=2cm,right=2cm,top=2.5cm,bottom=2.5cm]{geometry}

\setlength{\parindent}{0em}
\setlength{\parskip}{0.75em}

%%% Vars, re-use some texts
\newcommand{\pVersion}{v1.1}
\newcommand{\pTitle}{Minimalistic SDHC-SPI Host Reader}
%%%

%%% set header and footer
\usepackage{fancyhdr}
\pagestyle{fancy}
\lhead{ \pTitle \hspace{}  \pVersion}
\rhead{ Paulino Ruiz-de-Clavijo Vázquez }
\renewcommand{\footrulewidth}{0.4pt} % default is 0pt
\lfoot{Dept. Tecnología Electrónica, Universidad de Sevilla}
\cfoot{}
\rfoot{\thepage}



\renewcommand*{\arraystretch}{1.3}

\title{\pTitle}
\author{Paulino Ruiz-de-Clavijo V\'azquez \\ 
\emph{Departamento de Tecnología Electrónica, Universidad de Sevilla} }
\date{April, 2017 \pVersion}


\begin{document}

\maketitle

\section*{Overview}

\begin{figure}[h]
\centering
\includegraphics{figure-top-module}
\caption{Top view module}
\label{fig-top}
\end{figure}

MinSDHC-SPI Host is an IP core written in VHDL that implements the SD card 
protocol using Finite State Machines (FSM). It is has been designed to use a 
low resources but losing some performance compared with other solutions. 
The implementation uses SD SPI mode 
and a minimal set of SD commands to initialize and read SD cards.

\section*{Features}

The module performs two operations: SD card initialization and blocks read. 
SD card initialization is auto-triggered when the \emph{reset} signal
is de-asserted (set to 0). Read operation is controller by signals 
\emph{r\_block} and \emph{r\_byte}. This implementation uses the minimal set of 
CMD  commands, therefore the CMD12 is not used and the read process is aborted 
waiting for all block bytes. The read operation is performed in the single block 
mode.

Main features are:

\begin{itemize}
\item Only SDHC cards are supported
\item Programmable SPI pre-scaler
\item Picoblaze interface
\item Alternative architecture in VDHL for simulation purpose
\end{itemize}

The main purpose of this implementation is to achieve low footprint, 
sacrificing performance, the table \ref{tab-resources} show results for some
Xilinx's FPGAs.

\begin{table}[ht]
\renewcommand*{\arraystretch}{1.3}
\centering
\begin{tabular}{ | l c c c c | }
\hline
  \bfseries FPGA & \bfseries Slices &
  \bfseries Slices Reg. & \bfseries LUTs & \bfseries Slices (\%) \\
 \hline
  Spartan-3 - XC3S100E & 179 &  111 &  270 & 18.6\%  \\ 
  Virtex-5 - XC5VLX50T & 116 &  111 &  211 &  1.61\% \\ 
  Artix-7 - XC7A35T    &  63 &   88 &  211 &  0.77\% \\ 
 \hline
\end{tabular}
\caption{Resources utilization for some Xilinx's FPGAs}
\label{tab-resources}
\end{table}

\section*{Functional description and I/O signals}

The Figure \ref{fig-top} presents the I/O interface signals, 
they are described in Table \ref{tab_signals}. 


\begin{table}[H]
\centering
\begin{tabular}{ | l c l | } 
 \hline
  \bfseries Signal & \bfseries Direc. & \bfseries Description \\
  reset  & input & Reset / SD Card Initialization trigger  \\ %%
  clk50m & input & 50MHz input clock \\ %%
  addr[31:0] & input & 32-bit SD card block address to be read  \\
  r\_block & input & Reads a new block addressed by addr[31:0]  \\
  r\_byte &  input & Gets next byte of the current block \\
  dout[7:0] &  output & Output data bus \\
  busy & output & Operation running indicator \\
  err & output & Error condition indicator \\
  \hline  
\end{tabular}
\caption{Core I/O signals}
\label{tab_signals}
\end{table}

SD card initialization is auto-triggered when the \emph{reset} signal 
is de-asserted (set to 0). The Figure \ref{fig-init} contains a timing diagram 
with the initialization sequence.  In the Figure the \emph{busy} 
signal keeps asserted while the initialization process is not finished. This 
operation is completed when \emph{busy} signal falls, 
on error (card is not present or not supported) \emph{err} signal is asserted. 

\begin{figure}[H]
\centering
\includegraphics{figure-init}
\caption{Top view module}
\label{fig-init}
\end{figure}

Once the initialization process success, the read operation is performed 
setting the SD block number at \emph{addr} bus and asserting the signal 
\emph{r\_block}. The module behavior is similar to the initialization process,
while the block is not ready, the \emph{busy} signal keeps asserted. After
\emph{busy} fall, if \emph{err} is asserted an error happened otherwise,
the first byte of the 512-byte block is ready in \emph{dout} bus. The 
next 511 bytes are gotten asserting the \emph{r\_byte} signal 511 times as is
shown in the timing diagram of the Figure \ref{fig-readblock}.

The block read operation can be interrupted at any time by de-asserting the
\emph{r\_block} signal.

\begin{figure}[H]
\centering
\includegraphics{figure-readblock}
\caption{Read block operation}
\label{fig-readblock}
\end{figure}

The following examples shows how the module can be controlled from a FSM 
written in VHDL. The full VHDL code for the examples are is available in the 
examples directory.

The first example shows the initialization process controlled by another FSM 
using two states. The following VHDL code  starts resetting the module and wait 
until the SD card is ready or an error happens. 

\begin{minted}[frame=single]{vhdl}
    when ST_INIT_SD =>             -- Wait SDHOST INIT
      sdhost_reset <= '1';
      next_st <= ST_WAIT_SD_READY;

    when ST_WAIT_SD_READY =>       -- Wait for SDHOST ready
      sdhost_reset <= '0';
      if sdhost_busy= '1' then 
        next_st <= ST_WAIT_SD_READY;
      elsif sdhost_err='1' then 
        next_st <= ST_SD_INIT_ERR;
      else
        next_st <= ST_SD_INIT_DONE;
      end if;
\end{minted}

The next example is a one block read operation and it is achieved using three 
states. A counter is added to detect when the 512 bytes of the block are read.

\begin{minted}[frame=single]{vhdl}
     when ST_READ_BLOCK =>
       sdhost_r_block <= '1';     -- Read block
       next_st        <= ST_WAIT_BYTE;

     when ST_WAIT_BYTE =>         -- Wait for a byte ready
       sdhost_r_block <= '1';     -- Mandatory keep signal asserted
       if sdhost_busy= '1' then 
         next_st <= ST_WAIT_BYTE;
       elsif sdhost_err='1' then 
         next_st <= ST_ERR;
       else                      
         ...                      -- Byte ready, do something ...
         next_st <= ST_LOOP;
       end if;

      when ST_LOOP =>             -- Loop to read 512 bytes from SD card.
        sdhost_r_block  <= '1';   -- Keep asserted to get 
                                  -- other byte on same block
        if (byte_counter = b"111111111") then -- 512 bytes read
          next_st <= ST_END_BLOCK;
        else
          byte_counter_up <= '1'; -- byte_counter = byte_counter + 1;
          sdhost_r_byte   <= '1'; -- r_byte pulse to get other byte
          next_st         <= ST_WAIT_BYTE;
        end if;
\end{minted}


\section*{Picoblaze-3 interface}

An example with the module connected as a peripheral to Picoblaze-3 
(KCPSM3 \cite{picoblaze.2011}) 
microcontroller is supplied.  It allow read SD cards with smalls pieces of 
Picoblaze assembler code. Although an SD card can be controlled from Picoblaze 
adding an SPI peripheral, the code required spends a lot of program code due 
Picoblaze-3 has only 1024-instructions for program code available.

The file \emph{rtl/if\_picoblaze.vhdl} contains an example with
the glue logic for Picoblaze-3. The proposed logic uses one output port of 
Picoblaze and two  input ports.
Writing a byte in the out port, an operation of the table 
\ref{tab-pico-outport} is triggered. The operation 0x01 need a 4-byte argument,
the SD card block (32-bit number). This argument must be sent after 0x01 with 4 
writings in this port. The two input ports are selected by \emph{addr}.

This example has some issues like the absence of reset operation, but it can be 
done by adding the module reset signal to some port. It will improve in future 
version using the KCPSM6.



\begin{table}[htp]

\centering
\begin{tabular}{ | c l | }
\hline
  \bfseries Value & \bfseries Description \\
 \hline
  0x01 & Send the 32-bit address, after this byte 4 bytes more are expected with 
the SD card block address \\
  0x02 & Starts a read for the block address received with the command 0x01 \\
  0x04 & Reads the next byte of the current block \\
 \hline
\end{tabular}
\caption{Write port operation}
\label{tab-pico-outport}
\end{table}


\begin{table}[htp]
\centering
\begin{tabular}{ | c l | }
\hline
  \bfseries addr & \bfseries Description \\
 \hline
  0 &  Read status register\\
  1 &  Read SDHC-SPI byte out\\
 \hline
\end{tabular}
\caption{Read port operations}
\label{tab-read-operations}
\end{table}

\begin{table}[htp]

\centering
\begin{tabular}{ | c | c | c | c | c | c | c | c |}
\hline
  \bfseries bit 7 & \bfseries bit 6 & 
  \bfseries bit 5 & \bfseries bit 4 & \bfseries bit 3 &
  \bfseries bit 2 & \bfseries bit 1 & \bfseries bit 0 \\
 \hline
  - & - & - & - & - & 
  sdhost\_busy &  sdhost\_err & byte\_ready  \\
 \hline
\end{tabular}
\caption{Status register}
\label{tab-status-register}
\end{table}



\begin{thebibliography}{1}

\bibitem{picoblaze.2011}
Xilinx Inc., PicoBlaze 8-bit Embedded Microcontroller User Guide for 
Extended
  Spartan-3 and Virtex-5 FPGAs. Introducing PicoBlaze for Spartan-6,Virtex-6,
  and 7 Series FPGAs (2011).
 
\end{thebibliography}



\end{document}